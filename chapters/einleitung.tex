\section{Einleitung}
Im Rahmen dieses Projekts wurde eine professionelle Webseite für eine große Non-Profit-Organisation entwickelt, die sich mit den Herausforderungen des Klimawandels beschäftigt. Ziel der Webseite "Klima Transparenz" ist es, mehr Transparenz darüber zu schaffen, welche Unternehmen und Länder jährlich wie viel CO2 emittieren. Die Webseite soll der Öffentlichkeit zugänglich gemacht werden und eine zentrale Anlaufstelle bieten, um wichtige Informationen zu CO2-Emissionen auf eine leicht verständliche und ansprechend gestaltete Weise darzustellen.

Die Aufgabe bestand darin, eine benutzerfreundliche und responsive Webanwendung zu entwerfen und umzusetzen, die den hohen Ansprüchen der Transparenz gerecht wird und gleichzeitig ein modernes und konsistentes Design aufweist. Besondere Anforderungen an das Projekt umfassten die Integration einer sortier- und filterbaren Tabelle mit Emissionsdaten, die Absicherung aller Eingabefelder gegen mögliche Bedrohungen, sowie die Implementierung von mehreren Informationsseiten (z.B. Impressum, Datenschutz, Nutzungsbedingungen).

Diese Dokumentation beschreibt die Vorgehensweise bei der Entwicklung der Webseite, beginnend mit der Konzeption, über die technische Umsetzung, bis hin zu den getroffenen Sicherheitsmaßnahmen und der finalen Implementierung. Die Dokumentation wird dabei sowohl technische Aspekte als auch Designentscheidungen erläutern und reflektieren, um einen umfassenden Einblick in den Entwicklungsprozess zu geben.
\lipsum[1-2]
